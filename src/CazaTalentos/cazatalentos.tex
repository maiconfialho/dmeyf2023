% Options for packages loaded elsewhere
\PassOptionsToPackage{unicode}{hyperref}
\PassOptionsToPackage{hyphens}{url}
%
\documentclass[
]{article}
\usepackage{amsmath,amssymb}
\usepackage{iftex}
\ifPDFTeX
  \usepackage[T1]{fontenc}
  \usepackage[utf8]{inputenc}
  \usepackage{textcomp} % provide euro and other symbols
\else % if luatex or xetex
  \usepackage{unicode-math} % this also loads fontspec
  \defaultfontfeatures{Scale=MatchLowercase}
  \defaultfontfeatures[\rmfamily]{Ligatures=TeX,Scale=1}
\fi
\usepackage{lmodern}
\ifPDFTeX\else
  % xetex/luatex font selection
\fi
% Use upquote if available, for straight quotes in verbatim environments
\IfFileExists{upquote.sty}{\usepackage{upquote}}{}
\IfFileExists{microtype.sty}{% use microtype if available
  \usepackage[]{microtype}
  \UseMicrotypeSet[protrusion]{basicmath} % disable protrusion for tt fonts
}{}
\makeatletter
\@ifundefined{KOMAClassName}{% if non-KOMA class
  \IfFileExists{parskip.sty}{%
    \usepackage{parskip}
  }{% else
    \setlength{\parindent}{0pt}
    \setlength{\parskip}{6pt plus 2pt minus 1pt}}
}{% if KOMA class
  \KOMAoptions{parskip=half}}
\makeatother
\usepackage{xcolor}
\usepackage[margin=1in]{geometry}
\usepackage{color}
\usepackage{fancyvrb}
\newcommand{\VerbBar}{|}
\newcommand{\VERB}{\Verb[commandchars=\\\{\}]}
\DefineVerbatimEnvironment{Highlighting}{Verbatim}{commandchars=\\\{\}}
% Add ',fontsize=\small' for more characters per line
\usepackage{framed}
\definecolor{shadecolor}{RGB}{248,248,248}
\newenvironment{Shaded}{\begin{snugshade}}{\end{snugshade}}
\newcommand{\AlertTok}[1]{\textcolor[rgb]{0.94,0.16,0.16}{#1}}
\newcommand{\AnnotationTok}[1]{\textcolor[rgb]{0.56,0.35,0.01}{\textbf{\textit{#1}}}}
\newcommand{\AttributeTok}[1]{\textcolor[rgb]{0.13,0.29,0.53}{#1}}
\newcommand{\BaseNTok}[1]{\textcolor[rgb]{0.00,0.00,0.81}{#1}}
\newcommand{\BuiltInTok}[1]{#1}
\newcommand{\CharTok}[1]{\textcolor[rgb]{0.31,0.60,0.02}{#1}}
\newcommand{\CommentTok}[1]{\textcolor[rgb]{0.56,0.35,0.01}{\textit{#1}}}
\newcommand{\CommentVarTok}[1]{\textcolor[rgb]{0.56,0.35,0.01}{\textbf{\textit{#1}}}}
\newcommand{\ConstantTok}[1]{\textcolor[rgb]{0.56,0.35,0.01}{#1}}
\newcommand{\ControlFlowTok}[1]{\textcolor[rgb]{0.13,0.29,0.53}{\textbf{#1}}}
\newcommand{\DataTypeTok}[1]{\textcolor[rgb]{0.13,0.29,0.53}{#1}}
\newcommand{\DecValTok}[1]{\textcolor[rgb]{0.00,0.00,0.81}{#1}}
\newcommand{\DocumentationTok}[1]{\textcolor[rgb]{0.56,0.35,0.01}{\textbf{\textit{#1}}}}
\newcommand{\ErrorTok}[1]{\textcolor[rgb]{0.64,0.00,0.00}{\textbf{#1}}}
\newcommand{\ExtensionTok}[1]{#1}
\newcommand{\FloatTok}[1]{\textcolor[rgb]{0.00,0.00,0.81}{#1}}
\newcommand{\FunctionTok}[1]{\textcolor[rgb]{0.13,0.29,0.53}{\textbf{#1}}}
\newcommand{\ImportTok}[1]{#1}
\newcommand{\InformationTok}[1]{\textcolor[rgb]{0.56,0.35,0.01}{\textbf{\textit{#1}}}}
\newcommand{\KeywordTok}[1]{\textcolor[rgb]{0.13,0.29,0.53}{\textbf{#1}}}
\newcommand{\NormalTok}[1]{#1}
\newcommand{\OperatorTok}[1]{\textcolor[rgb]{0.81,0.36,0.00}{\textbf{#1}}}
\newcommand{\OtherTok}[1]{\textcolor[rgb]{0.56,0.35,0.01}{#1}}
\newcommand{\PreprocessorTok}[1]{\textcolor[rgb]{0.56,0.35,0.01}{\textit{#1}}}
\newcommand{\RegionMarkerTok}[1]{#1}
\newcommand{\SpecialCharTok}[1]{\textcolor[rgb]{0.81,0.36,0.00}{\textbf{#1}}}
\newcommand{\SpecialStringTok}[1]{\textcolor[rgb]{0.31,0.60,0.02}{#1}}
\newcommand{\StringTok}[1]{\textcolor[rgb]{0.31,0.60,0.02}{#1}}
\newcommand{\VariableTok}[1]{\textcolor[rgb]{0.00,0.00,0.00}{#1}}
\newcommand{\VerbatimStringTok}[1]{\textcolor[rgb]{0.31,0.60,0.02}{#1}}
\newcommand{\WarningTok}[1]{\textcolor[rgb]{0.56,0.35,0.01}{\textbf{\textit{#1}}}}
\usepackage{graphicx}
\makeatletter
\def\maxwidth{\ifdim\Gin@nat@width>\linewidth\linewidth\else\Gin@nat@width\fi}
\def\maxheight{\ifdim\Gin@nat@height>\textheight\textheight\else\Gin@nat@height\fi}
\makeatother
% Scale images if necessary, so that they will not overflow the page
% margins by default, and it is still possible to overwrite the defaults
% using explicit options in \includegraphics[width, height, ...]{}
\setkeys{Gin}{width=\maxwidth,height=\maxheight,keepaspectratio}
% Set default figure placement to htbp
\makeatletter
\def\fps@figure{htbp}
\makeatother
\setlength{\emergencystretch}{3em} % prevent overfull lines
\providecommand{\tightlist}{%
  \setlength{\itemsep}{0pt}\setlength{\parskip}{0pt}}
\setcounter{secnumdepth}{-\maxdimen} % remove section numbering
\ifLuaTeX
  \usepackage{selnolig}  % disable illegal ligatures
\fi
\IfFileExists{bookmark.sty}{\usepackage{bookmark}}{\usepackage{hyperref}}
\IfFileExists{xurl.sty}{\usepackage{xurl}}{} % add URL line breaks if available
\urlstyle{same}
\hypersetup{
  pdftitle={Cazatalentos},
  pdfauthor={Maicon Araújo Fialho},
  hidelinks,
  pdfcreator={LaTeX via pandoc}}

\title{Cazatalentos}
\author{Maicon Araújo Fialho}
\date{2023-09-10}

\begin{document}
\maketitle

\begin{Shaded}
\begin{Highlighting}[]
\NormalTok{df\_cazatalentos5 }\OtherTok{\textless{}{-}} \FunctionTok{data.frame}\NormalTok{(}
  \AttributeTok{Cazatalentos =} \FunctionTok{rep}\NormalTok{(}\StringTok{"Cazatalentos 5"}\NormalTok{, }\AttributeTok{times =} \DecValTok{5}\NormalTok{),}
  \AttributeTok{Candidata =} \FunctionTok{c}\NormalTok{(}\StringTok{"A"}\NormalTok{, }\StringTok{"B"}\NormalTok{, }\StringTok{"C"}\NormalTok{, }\StringTok{"D"}\NormalTok{, }\StringTok{"E"}\NormalTok{),}
  \AttributeTok{Aciertos1 =} \FunctionTok{c}\NormalTok{(}\DecValTok{85}\NormalTok{, }\DecValTok{84}\NormalTok{, }\DecValTok{84}\NormalTok{, }\DecValTok{82}\NormalTok{, }\DecValTok{81}\NormalTok{),}
  \AttributeTok{Aciertos2 =} \FunctionTok{c}\NormalTok{(}\DecValTok{69}\NormalTok{, }\DecValTok{74}\NormalTok{, }\DecValTok{74}\NormalTok{, }\DecValTok{70}\NormalTok{, }\DecValTok{75}\NormalTok{),}
  \AttributeTok{Aciertos3 =} \FunctionTok{c}\NormalTok{(}\DecValTok{70}\NormalTok{, }\DecValTok{76}\NormalTok{, }\DecValTok{75}\NormalTok{, }\DecValTok{73}\NormalTok{, }\DecValTok{74}\NormalTok{)}
\NormalTok{)}

\NormalTok{df\_cazatalentos9 }\OtherTok{\textless{}{-}} \FunctionTok{data.frame}\NormalTok{(}
  \AttributeTok{Cazatalentos =} \FunctionTok{rep}\NormalTok{(}\StringTok{"Cazatalentos 9"}\NormalTok{, }\AttributeTok{times =} \DecValTok{1}\NormalTok{),}
  \AttributeTok{Candidata =} \FunctionTok{c}\NormalTok{(}\StringTok{"A"}\NormalTok{),}
  \AttributeTok{Aciertos =} \FunctionTok{c}\NormalTok{(}\DecValTok{68}\NormalTok{, }\DecValTok{74}\NormalTok{, }\DecValTok{78}\NormalTok{, }\DecValTok{70}\NormalTok{, }\DecValTok{68}\NormalTok{, }\DecValTok{63}\NormalTok{, }\DecValTok{80}\NormalTok{, }\DecValTok{68}\NormalTok{, }\DecValTok{67}\NormalTok{, }\DecValTok{65}\NormalTok{)}
\NormalTok{)}
\end{Highlighting}
\end{Shaded}

\hypertarget{apenas-a-cazatalentos5-disponibilizou-dados-para-estudarmos}{%
\subsection{Apenas a cazatalentos5 disponibilizou dados para
estudarmos}\label{apenas-a-cazatalentos5-disponibilizou-dados-para-estudarmos}}

\hypertarget{calcular-a-muxe9dia-e-a-variuxe2ncia-dos-jogadores-do-cazatalentos-5}{%
\subsubsection{Calcular a média e a variância dos jogadores do
Cazatalentos
5}\label{calcular-a-muxe9dia-e-a-variuxe2ncia-dos-jogadores-do-cazatalentos-5}}

\begin{Shaded}
\begin{Highlighting}[]
\NormalTok{media\_jogadores5 }\OtherTok{\textless{}{-}} \FunctionTok{rowMeans}\NormalTok{(df\_cazatalentos5[, }\DecValTok{3}\SpecialCharTok{:}\DecValTok{5}\NormalTok{], }\AttributeTok{na.rm =} \ConstantTok{TRUE}\NormalTok{)}
\NormalTok{variancia\_jogadores5 }\OtherTok{\textless{}{-}} \FunctionTok{apply}\NormalTok{(df\_cazatalentos5[, }\DecValTok{3}\SpecialCharTok{:}\DecValTok{5}\NormalTok{], }\DecValTok{1}\NormalTok{, var, }\AttributeTok{na.rm =} \ConstantTok{TRUE}\NormalTok{)}
\end{Highlighting}
\end{Shaded}

\hypertarget{exibir-os-resultados}{%
\subsubsection{Exibir os resultados}\label{exibir-os-resultados}}

Média dos jogadores do Cazatalentos 5:

\begin{Shaded}
\begin{Highlighting}[]
\FunctionTok{print}\NormalTok{(media\_jogadores5)}
\end{Highlighting}
\end{Shaded}

\begin{verbatim}
## [1] 74.66667 78.00000 77.66667 75.00000 76.66667
\end{verbatim}

Variância dos jogadores do Cazatalentos 5:

\begin{Shaded}
\begin{Highlighting}[]
\FunctionTok{print}\NormalTok{(variancia\_jogadores5)}
\end{Highlighting}
\end{Shaded}

\begin{verbatim}
## [1] 80.33333 28.00000 30.33333 39.00000 14.33333
\end{verbatim}

\hypertarget{conclusion}{%
\section{Conclusion}\label{conclusion}}

\end{document}
